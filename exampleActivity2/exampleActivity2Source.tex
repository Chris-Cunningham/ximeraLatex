\shortdescription{In this activity we give a second example.} 
\activitytitle{Second Example}
\prerequisites{none}
\outcomes{ximeraLatex}


Here we have a multipart question with free-response.

\begin{question} 
Suppose you are standing on a bridge that is 60 meters above
sea-level. You toss a ball up into the air with an initial velocity of
30 meters per second.  If $t$ is the time (in seconds) after we toss
the ball, then the height at time $t$ is approximately $f(t) = -5 t^2
+30t+60$. What does $f(2)$ mean in our context?
\begin{solution}
\begin{hint}
We want an answer in the context of the problem. 
\end{hint}
\answer[free-response]{The value $f(2)$ is the height of the ball after $2$ seconds.}
\end{solution}
Now suppose $t$ is such that $f(t) = 100$. What does this mean in our
context?
\begin{solution}
\begin{hint}
We want an answer in the context of the problem. 
\end{hint}
\answer[free-response]{These value of $t$ are the times when the ball is at 100 meters above sea level.}
\end{solution}
Finally, if $h$ is a small positive value what is the meaning of
$f(t+h)$? How does this compare to the meaning of $f(t)+h$?
\begin{solution}
\begin{hint}
We want an answer in the context of the problem. 
\end{hint}
\answer[free-response]{The value $f(t+h)$ gives the height of the ball
  slightly after time $t$. On the other hand, the value $f(t)+h$ gives
  a height just higher than the ball at time $t$.}
\end{solution}
\end{question}

If you like, check out this video\youtube{http://www.youtube.com/watch?v=0aQpLSu2fMs}.
