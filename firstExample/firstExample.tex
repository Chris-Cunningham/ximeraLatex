\documentclass{ximera}
%% handout
%% space
%% newpage
%% numbers
%% nooutcomes

%% You can put user macros here
%% However, you cannot make new environments

\graphicspath{{./}{firstExample/}}

\usepackage{tikz}
\usepackage{tkz-euclide}
\usetkzobj{all}

\tikzstyle geometryDiagrams=[ultra thick,color=blue!50!black]
 %% we can turn off input when making a master document

\outcome{Understand a first example of the Ximera style.}
\outcome{Have a nice basic example to work from.}
 
\title{First example}

\begin{document}
\begin{abstract}
In this activity we see some examples.
\end{abstract}
\maketitle

To start we can have theorem environments:

\begin{theorem}
Given a right triangle drawn with TiKZ:
\begin{image}
\begin{tikzpicture}[geometryDiagrams]
\coordinate (A) at (0,2);
\coordinate (B) at (0,5);
\coordinate (C) at (6.5,2);
\tkzMarkRightAngle(C,A,B)
\tkzDefMidPoint(A,B) \tkzGetPoint{a}
\tkzDefMidPoint(A,C) \tkzGetPoint{b}
\tkzDefMidPoint(B,C) \tkzGetPoint{c}
\draw (A)--(B)--(C)--cycle;
\tkzLabelPoints[above](c)
\tkzLabelPoints[below](b)
\tkzLabelPoints[left](a)
\end{tikzpicture}
\end{image}
We have that:
\[
a^2 + b^2 = c^2
\]
\end{theorem}


As well as example environments.

\begin{example}
For example, this is an example.
\end{example}


There are exercises you can do:

\begin{exercise}
$3\times 2=\answer{6}$
\end{exercise}

Some exercises can have hints.

\begin{exercise}
Given that $r(v)=-2 v^2-4 v-4$, evaluate $r(-0.4)$. Express your answer in decimal notation.
%
\begin{hint}
$r(-0.4)=-2 (-0.4)^2-4 (-0.4)-4$.
\end{hint}
\begin{hint}
$r(-0.4)=-2.72$.
\end{hint}

The value of the function $r(v)=-2 v^2-4 v-4$, evaluated at $v=-0.4$, is $\answer{-2.72}$.
%
\end{exercise}



\begin{question}
What is the worst kind of cat?
\begin{prompt}
\begin{multipleChoice}
\choice{tabby}
\choice[correct]{puppy}
\choice{dog}
\choice{kitten}
\choice{main coon}
\end{multipleChoice}
\end{prompt}
\begin{hint}
It is not a cat or a type of cat.
\end{hint}
\begin{hint}
It is a puppy!
\end{hint}
\end{question}

It is also possible to have a list of questions which get shuffled:


\begin{shuffle}
\begin{question}
In the plot below, is $P$ a function of $k$?
\begin{image}
\begin{tikzpicture}
\begin{axis}[
            ymin=-5,
			ymax=5,
            axis lines =center, xlabel=$k$, ylabel=$P$,
              every axis y label/.style={at=(current axis.above origin),anchor=south},
              every axis x label/.style={at=(current axis.right of origin),anchor=west},
            domain=-5:5,
            grid = major,
            xtick={-4,...,4},
            ytick={-4,...,4},
          ]
          \addplot [very thick, smooth] {1 + (3.5 + x)*(-0.5714285714285714 + (-3.5 + x)*(0.16326530612244897 + (-0.3327149041434756 + (-0.20522334808049095 + 0.04019472590901159*(-3 + x))*(-2 + x))*x))};
        \end{axis}
\end{tikzpicture}
\end{image}

\begin{multipleChoice}
\choice[correct]{Yes.}
\choice{No.}
\end{multipleChoice}
\begin{hint}
For each input, how many outputs are there?
\end{hint}

Use the plot to compute $P(2)$.

\begin{hint}
To start, find $2$ on the horizontal axis. 
\end{hint}
\begin{hint}
Now from this position, move up or down until you reach the curve. The value of $P(2)$ is the height of the curve at the point $k=2$.
\end{hint}
The value of $P(2)$ is \answer{$2$}.
\end{question}


\begin{question}
In the plot below, is $R$ a function of $n$?
\begin{image}
\begin{tikzpicture}
\begin{axis}[
            ymin=-5,
			ymax=5,
            axis lines =center, xlabel=$n$, ylabel=$R$,
              every axis y label/.style={at=(current axis.above origin),anchor=south},
              every axis x label/.style={at=(current axis.right of origin),anchor=west},
            domain=-5:5,
            grid = major,
            xtick={-4,...,4},
            ytick={-4,...,4},
          ]
          \addplot [very thick, smooth] {4 + (-0.42857142857142855 + (-0.05952380952380952 + (0.09163059163059163 + (-0.041447441447441453 - 0.08955488955488956*(-3 + x))*(-2 + x))*(-0.5 + x))*(-3.5 + x))*(3.5 + x)};
        \end{axis}
\end{tikzpicture}
\end{image}

\begin{multipleChoice}
\choice[correct]{Yes.}
\choice{No.}
\end{multipleChoice}
\begin{hint}
For each input, how many outputs are there?
\end{hint}

Use the plot to compute $R(3)$.

\begin{hint}
To start, find $3$ on the horizontal axis. 
\end{hint}
\begin{hint}
Now from this position, move up or down until you reach the curve. The value of $R(3)$ is the height of the curve at the point $n=3$.
\end{hint}
The value of $R(3)$ is $\answer{1}$.
\end{question}

\begin{question}
In the plot below, is $b$ a function of $w$?
\begin{image}
\begin{tikzpicture}
\begin{axis}[
            ymin=-5,
			ymax=5,
            axis lines =center, xlabel=$w$, ylabel=$b$,
              every axis y label/.style={at=(current axis.above origin),anchor=south},
              every axis x label/.style={at=(current axis.right of origin),anchor=west},
            domain=-5:5,
            grid = major,
            xtick={-4,...,4},
            ytick={-4,...,4},
          ]
          \addplot [very thick, smooth] {2 + (3.5 + x)*(0.2857142857142857 + (-3.5 + x)*(0.489795918367347 + x*(0.34013605442176864 + (-0.5850340136054422 - 0.25117739403453676*(-0.5 + x))*(2 + x))))};
        \end{axis}
\end{tikzpicture}
\end{image}

\begin{multipleChoice}
\choice[correct]{Yes.}
\choice{No.}
\end{multipleChoice}
\begin{hint}
For each input, how many outputs are there?
\end{hint}

Use the plot to compute $b(-2)$.

\begin{hint}
To start, find $-2$ on the horizontal axis. 
\end{hint}
\begin{hint}
Now from this position, move up or down until you reach the curve. The value of $b(-2)$ is the height of the curve at the point $w=-2$.
\end{hint}
The value of $b(-2)$ is $\answer{4}$.

\end{question}
\end{shuffle}

\begin{question}
Enter the matrix  \(\begin{bmatrix} x & y \\ xy & z+1 \end{bmatrix}\)
\begin{matrixAnswer}
  correctMatrix = [['x','y'],['xy','z+1']]
\end{matrixAnswer}
\end{question}



\end{document}
